\documentclass[12pt,a4paper]{article}

\usepackage[utf8]{inputenc}
\usepackage{enumitem}
\usepackage{graphicx}
\usepackage{titlepic}
\usepackage{textcomp}
\usepackage{float}
\usepackage{caption}

\renewcommand{\figurename}{Fig.}
\renewcommand{\contentsname}{Índice}

\newcommand{\tab}{\hspace{1cm}}
\newcommand{\data}{\today}

\setlist[description]{leftmargin=\parindent,labelindent=\parindent,rightmargin=\parindent}

\captionsetup[table]{skip=10pt}

\begin{document}
\pagenumbering{gobble}

\begin{titlepage}
\begin{center}
  \vspace*{1cm}

  {\Huge \textbf{Programação Avançada}} \\
  \vspace{0.8cm}
  \textbf{Meta 2}
  \vspace{0.8cm}

  \vspace{9.0cm}

  \textbf{André Pedro \\ a21240337@alunos.isec.pt}
  \\[0.6cm]
  \textbf{Romeu Gomes \\ a21240332@alunos.isec.pt}
  \\[1cm]
  \textbf{\data{}}

  \vfill
  \textit{ISEC - Instituto Superior de Engenharia de Coimbra\\
  Departamento de Engenharia Informática e Sistemas\\}

\end{center}
\end{titlepage}

\pagenumbering{arabic}

\tableofcontents
\newpage\noindent

\section{Diagrama da Máquina de Estados}
\begin{figure}[!ht]
  \centering
    \includegraphics[width=1\textwidth]{Main.jpg}
\end{figure}

\newpage\noindent

\section{Descrição das Classes}
\subsection{Classes no Package \textit{logicaJogo}}

\begin{enumerate}
\item Jogo - Classe onde existem métodos para alterar o Estado actual do jogo. É usado ao longo do jogo para alterar os estados e fazer o jogador progredir.
\item Utils - Classe estática com métodos que possam ser úteis, como, por exemplo, o lançamento do dado.
\end{enumerate}

\newpage\noindent

\subsection{Classes no Package \textit{logicaJogo.Dados}}

\begin{enumerate}

\item DadosJogo - Classe que irá conter toda a informação do jogo, como por exemplo, o número de pontos de riqueza, metal, vitória, etc. Também inclui uma grande quantidade de funcões que servem para ajudar outras classes a obter informações relativas ao estado do jogo.
\end{enumerate}

\newpage\noindent

\subsection{Classes no Package \textit{logicaJogo.Eventos}}

\begin{enumerate}
\item Evento - Classe que serve de base para todas as outras classes que representam eventos.
\item Asteroid - Classe que corresponde ao evento \textit{Asteroide} e tem o método principal \textit{execute}. Adiciona o recurso indicado na carta.
\item DerelictShip - Classe que corresponde ao evento \textit{Nave Abandonada} e tem o método principal \textit{execute}. Adiciona o recurso indicado na carta.
\item LargeInvasionForce - Classe que corresponde ao evento \textit{Invasão Grande} e tem o método principal \textit{execute}. É uma invasão superior à pequena.
\item PeaceAndQuiet - Classe que corresponde ao evento \textit{Sem Evento} e tem o método principal \textit{execute}. Não acontece nada.
\item Revolt - Classe que corresponde ao evento \textit{Revolta} e tem o método principal \textit{execute}. Pode tornar um sistema não conquistado num sistema não alinhado, e, se as condições forem ideais, o jogador também pode perder o jogo.
\item SmallInvasionForce - Classe que corresponde ao evento \textit{Invasão Pequena} e tem o método principal \textit{execute}.
\item SRevolt - Classe que corresponde ao evento \textit{Revolta Pequena} e tem o método principal \textit{execute}. Tem um menor efeito do que a revolta normal.
\item Strike - Classe que corresponde ao evento \textit{Greve} e tem o método principal \textit{execute}. O sistema em greve deixa de oferecer recursos.
\end{enumerate}

\newpage\noindent

\subsection{Classes no Package \textit{logicaJogo.Sistemas}}

\begin{enumerate}
\item MSESystem - Classe base para ambas as classes NearSystem e DistantSystem. Contém toda a informação de um sistema.
\item NearSystem - Classe que representa os Sistemas que não são distantes.
\item DistantSystem - Classe que representa os Sistemas Distantes.
\end{enumerate}

\newpage\noindent

\subsection{Classes no Package \textit{logicaJogo.States}}

\begin{enumerate}
\item IStates - Interface onde estão definidos todos os estados.
\item StateAdapter - Classe onde se devolve cada estado e onde está a variável que contém os dados do jogo, para ser usada em diferentes situações.
\item PreparacaoJogo - Classe que representa o Estado do início do jogo, onde são definidas várias variáveis.
\item ExplorarAtacarConquistarPassar - Classe que representa o Estado em que se pode explorar, atacar ou conquistar Sistemas. É possível também passar esta fase sem fazer qualquer acção.
\item Trocar - Classe que representa o Estado em que o jogador tem a possibilidade de trocar vários recursos entre si, se tiver a tecnologia correcta activa.
\item MilitarTecnologia - Classe que representa o Estado em que o jogador tem a possibilidade, se tiver os recursos, de aumentar o seu nível militar ou comprar uma tecnologia.
\item FimJogo - Classe que representa o Estado do fim do jogo. O utilizador tem a possibilidade de sair do jogo, ou recomeçar um novo.
\end{enumerate}

\newpage\noindent

\subsection{Classes no Package \textit{logicaJogo.Tecnologias}}

\begin{enumerate}
\item Tecnologia - Classe que serve de classe base para todas as classes das tecnologias existentes.
\item ForwardStarbases - Classe que representa a tecnologia que permite explorar e conquistar Sistemas Distantes.
\item HyperTelevision - Classe que representa a tecnologia que aumenta a resistência durante um evento de revolta.
\item InterspecieCommerce - Classe que representa a tecnologia que permite fazer trocas.
\item InterstellarBanking - Classe que representa a tecnologia que permite aumentar a quantidade de recursos que é possível ter.
\item InterstellarDiplomacy - Classe que representa a tecnologia que permite explorar/atacar/conquistar um Sistema com sucesso sem qualquer tipo de requerimentos.
\item PlanetaryDefense - Classe que representa a tecnologia que permite aumentar a resistência durante um evento de Invasão.
\item RobotWorkers - Classe que representa uma tecnologia que permite obter metade dos recursos durante um evento de Greve.
\item CapitalShips - Classe que representa a tecnologia que permite o aumento da força militar de 3 para 5.
\end{enumerate}

\newpage\noindent

\subsection{Classes no Package \textit{ui.texto}}

\begin{enumerate}
\item IUTexto - Classe que contém todos os métodos de interacção com o utilizador. É a classe onde é definido o estado para que o utilizador vai, de acordo com a situação actual do jogador, e também dependendo das suas escolhas.
\end{enumerate}

\newpage\noindent

\subsection{Classes no Package \textit{ui.grafico.Controllers}}
\begin{enumerate}
\item MainController - Classe que representa o controlador da janela principal.
\item InitController - Classe que representa o controlador da janela inicial.
\item InfoTecController - Classe que representa o controlador da janela da explicação das tecnologias.
\item FinalController - Classe que representa o controlador da janela final do jogo.
\end{enumerate}
\newpage\noindent
    
\subsection{Classes no Package \textit{ui.grafico.Models}}
\begin{enumerate}
\item FinalModel - Classe que representa o modelo da janela final.
\item MainModel - Classe que representa o modelo da janela principal.
\item InitModel - Classe que representa o modelo da janela inicial.
\end{enumerate}
\newpage\noindent
    
\subsection{Classes no Package \textit{ui.grafico.Views}}
\begin{enumerate}
\item FinalView - Classe que representa a vista da janela final.
\item InfoTecView - Classe que representa a vista da janela da explicação das tecnologias.
\item InitView - Classe que representa a vista da janela inicial.
\item MainView - Classe que representa a vista da janela principal.
\end{enumerate}

\newpage\noindent

\section{Características não Implementadas}
\-\hspace{1cm}A única característica não implementada é a finalização do jogo quando o jogador está no segundo ano e acontece o evento \textit{Invasão}.

\end{document}